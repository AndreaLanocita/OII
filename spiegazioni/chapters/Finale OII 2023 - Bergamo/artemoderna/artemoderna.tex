% chapters/Finale OII 2023 - Bergamo/artemoderna.tex
%! TeX root = ../../../soluzioni.tex
\section{Vetrate Colorate - artemoderna}

\subsection*{Subtask 2}
E' dimostrabile che esista sempre una soluzione al problema.
Infatti:
\begin{itemize}
    \item Se l'ordine fosse decrescente, basterebbe ruotare insieme l'intera vetrata
    \item Se l'ordine fosse crescente, si potrebbe ruotare singolarmente ciascun vetro
\end{itemize}

\lstinputlisting[language=c++]{chapters/Finale OII 2023 - Bergamo/artemoderna/subtask2.cpp}

\subsection*{Subtask 3}
L'idea generale prevede di dividere le vetrate in segmenti $[a, b]$ che, girati insieme, rendano la sottosequenza $[0, b]$ ordinata.
\begin{prop}
    Sia data una sequenza di 0 ed 1 da ordinare in ordine crescente dividendola in sottosequenze contigue e invertendo ciascuna.
    Se tale ordinamento esiste, è ottenibile dividendo la sequenza dopo ciascuno 0 seguito da un 1.
\end{prop}
\begin{proof}
    Assumiamo che esista l'ordinamento cercato e applichiamo l'algoritmo descritto. Sia $[0, a]$ una delle sottosequenze ottenute.
    Allora:
    \begin{itemize}
        \item La sottosequenza è formata esclusivamente da zeri
        \item La sottosequenza presenta $n$ 1, nelle posizioni $[0, n-1]$
    \end{itemize}
    In ogni altro caso infatti, la sottosequenza conterrebbe uno 0 seguito da un 1, che è impossibile per costruzione.
    È facile vedere come in ciascuno dei due casi possibili, invertire la sottosequenza porti al suo ordinamento. \\
    Ora è necessario dimostrare come, nel caso in cui con questo algoritmo non ottenessimo una sequenza ordinata, l'ordinamento cercato non esista.
    Per assurdo, non dividiamo la sequenza dove è presente uno 0 con successivamente un 1, ma in una posizione successiva.
    Dopo l'inversione della sottostringa, otterremmo per forza un 1 seguito da uno 0, quindi l'ordinamento non sarebbe corretto. 
    Di conseguenza, l'unico algoritmo che può portare ad un risultato valido è quello descritto.
\end{proof}

Nel problema è specificato come tutta la vetrata vada ruotata, quindi bisogna ricordarsi di ruotare
la sottosequenza rimasta una volta arrivati alla fine dell'array.

\lstinputlisting[language=c++]{chapters/Finale OII 2023 - Bergamo/artemoderna/subtask3.cpp}

\subsection*{Soluzione Ottimale}
È possibile estendere la proposizione precedente al caso generale, per cui la divisione in sottosequenze
va effettuata tra $(a, b)$, dovunque $a<b$. 
\lstinputlisting[language=c++]{chapters/Finale OII 2023 - Bergamo/artemoderna/artemoderna.cpp}